\documentclass{xjtureport}
\usepackage{color}
\usepackage[colorlinks,linkcolor=blue]{hyperref}
\usepackage{booktabs}
\usepackage{multirow}
% \usepackage[table,xcdraw]{xcolor}
% \usepackage{enumerate}
% \newCJKfontfamily\sonti{FZCuSong-B09S}[BoldFont=FZCuSong-B09S]
% =============================================
% Part 0 Edit the info
% =============================================
% \titleformat{\section}{\bfseries\sonti\sectionef}{\thesection}{1em}
\major{大数据管理与应用}
\name{张锦羽}
\title{理性预期对金融市场有效性的理解与CAPM模型分析我国A股市场}
\stuid{2204112387}
% \college{管理学院}
\date{\today}
% \lab{寝室}
\course{现代金融学}
\instructor{李丽芳}
% \grades{59}
\expname{现代金融学期末大作业}% 副标题
% \exptype{设计实验}
% \partner{Bob}
% \setbeamertemplate{footline}[page number]
\setlength{\abovecaptionskip}{0.cm}
\setCJKmainfont{simsun.ttc}[ BoldFont=simhei.ttf,ItalicFont=simkai.ttf]
\begin{document}
% =============================================
% Part 1 Header
% ============================================={不想要封面就注释掉}============
\makecover
% \makeheader
\zihao{-4}

% =============================================
% Part 2 Main document
% =============================================
% \maketitle
\thispagestyle{fancy}

\rfoot{\thepage}
\tableofcontents
\clearpage
% 同学们好,我们期末考试采取大作业的形式,分两部分:
% 1. 500字左右谈一下你对理性预期与金融市场有效性的理解。(30分)
% 2. 运用CAPM模型分析一下我国A股市场是否有效。(60分)

% 作业没有标准答案,我更看重分析过程。可用中文或者英文答题,不允许抄袭,一旦发现期末成绩计0分。

% 作业提交截止时间:12月25日晚上11点
% \section{理性预期对有效市场的作用机制}
\section{理性预期与金融市场有效性分析}
\subsection{理性预期理论分析}
理性预期理论认为,金融市场上的投资者都是完全理性的,可以最好地利用所有可以获得的信息(从概率的角度,我们一般称之为先验信息),包括股票历史信息、政府政策信息等来形成自己的投资预期。也即,理性预期理论有以下三点含义:
\begin{enumerate}
    \item 作出经济决策的经济主体是有理性的。
    \item 主体在作出决策前会力图获得一切有关信息。
    \item 在决策时不会犯系统性错误,即使犯错误,也会及时有效地调整,使得在长期而言保持正确。
\end{enumerate}
\par 从理性预期理论的定义我们不难看出,投资预期是一个随机变量。它可能与最优值相等,也可能与最优值之间存在较大偏差
% \begin{itemize}
%     \item \textbf{两者的关系:}有效市场理论和理性预期假说是现代金融学理论的最重要基石,有效市场的假说是建立在市场中投资者完全理性和信息传播速度迅速的基础上的。
%     \item \textbf{理性预期理论分析}\\
\subsection{有效市场理论分析}
有效市场市场理论是理性预期理论在金融市场的一种应用。有以下三种形式:
    \begin{enumerate}
        \item 弱形式:即证券的现行价格仅充分反映了过去的一切公开信息(如历史数据)
        \item 半强假设:证券的现行价格不但反映了过去的一切公开信息,还反映了现在的一切公开信息
        \item 强假设:证券的价格不仅反映了过去、现在的一切公开信息,还充分反映了内部信息。
    \end{enumerate}
\par 根据以上特点我们可以得知:如果所有的信息都反映到了价格上,那么则不会存在套利的机会,任何投资者都不可能获得超过平均收益水平的额外收益,其获得的收益是对风险和机会成本的补偿。
\subsection{理性预期对有效市场的作用机制分析}
什么是有效的市场,根据我的理解与一些论文的分析\cite{},合理健康的市场是无法用技术手段作预测的,收益曲线符合随机游走的规律。因为现代股票分析技术,人工智能,深度学习方法都是基于先验数据构建模型进行训练得到结果。而根据理性预期理论,这些所有的先验信息都被市场中的投资者所熟知,从而辅助投资者作出决策。说的更加生动一点,理性预期理论认为市场中的消费者,人人都是“投资高手”,能够充分利用到自己掌握的所有历史,现在和内部的信息辅助进行投资决策。
\par 首先市场中的投资者基于先验的信息和能够收集到的所有信息,进行投资决策,目的是为了使得消费者的期望效用得到最大(基于消费者理性人的假设)。结合消费者自身对风险的偏好,消费者群体作出决策。最终群体决策的结果反映到证券市场的价格上,价格反映了消费者当前能够掌握的所有信息。
\par 证券的价格应该准确地反映收集到的所有关于未来定价的新资料和信息,这是基于证券市场的强有效性假设和投资者全部理性预期所形成的市场机制所作出的论断。因为一旦证券价格没有很迅速的变化到反映所有已知信息的水平,就存在掌握较多信息的人利用时间与信息差进行套利,使得证券价格很快维持到市场预期水准。但在理性预期和强有效性假设的前提下,市场上是不可能存在连续套利机会的。
\par 此外,我认为在理性预期假设和强有效性市场假设的前提下,“掌握较多信息的人”是一种伪命题。由理性预期理论可知,他们只是通过消除未被利用到的盈利机会而使得市场更具有效率的人。在理性预期假设中,信息的获取是无成本的,投资者也都是智力水平类似,分析能力类似的一大类群体,可以认为市场上投资者们的信息是共享的。因此“掌握较多信息的人”即使真的存在,也会因为市场上信息流动的迅速,最终迅速反映在证券价格中,不会存在持续性套利的空间。
\par 在理性预期均衡下,交易者没有重新订约的意愿,均衡处于稳定状态。它表示市场结清的均衡价格已不能为交易者提供新的可以利用的私人信息,也表示交易者已从均衡价格中窥探到其他交易者的所有私人信息,在这两种意义上理性预期均衡完全揭示私人信息,市场没有用新信息获取赢利的可能。
\par 总结,我认为理性预期假设对市场有效性的机制如下:
\begin{figure}[H]
    \begin{center}
        \includegraphics[width=0.8\textwidth]{figure/期望与市场有效性.png}
    \end{center}
    \caption{理性预期理论对市场有效性的作用机制概念图}
\end{figure}
\clearpage
\section{运用CAPM模型对我国A股市场有效性的分析}
\subsection{CAPM模型的综述}

\end{document}