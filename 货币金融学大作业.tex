\documentclass{xjtureport}
\usepackage{color}
\usepackage[colorlinks,linkcolor=blue]{hyperref}
\usepackage{booktabs}
\usepackage{multirow}
% \usepackage[table,xcdraw]{xcolor}
% \usepackage{enumerate}
% \newCJKfontfamily\sonti{FZCuSong-B09S}[BoldFont=FZCuSong-B09S]
% =============================================
% Part 0 Edit the info
% =============================================
% \titleformat{\section}{\bfseries\sonti\sectionef}{\thesection}{1em}
\major{大数据管理与应用}
\name{张锦羽}
\title{理性预期对金融市场有效性的理解与CAPM模型分析我国A股市场}
\stuid{2204112387}
\college{管理学院}
\date{\today}
% \lab{寝室}
\course{现代金融学}
\instructor{李丽芳}
% \grades{59}
\expname{现代金融学期末大作业}% 副标题
% \exptype{设计实验}
% \partner{Bob}
% \setbeamertemplate{footline}[page number]
\setlength{\abovecaptionskip}{0.cm}
\setCJKmainfont{simsun.ttc}[ BoldFont=simhei.ttf,ItalicFont=simkai.ttf]
\begin{document}
% =============================================
% Part 1 Header
% ============================================={不想要封面就注释掉}============
\makecover
% \makeheader
\zihao{-4}

% =============================================
% Part 2 Main document
% =============================================
% \maketitle
\thispagestyle{fancy}

\rfoot{\thepage}
\tableofcontents
\clearpage
% 同学们好,我们期末考试采取大作业的形式,分两部分:
% 1. 500字左右谈一下你对理性预期与金融市场有效性的理解。(30分)
% 2. 运用CAPM模型分析一下我国A股市场是否有效。(60分)

% 作业没有标准答案,我更看重分析过程。可用中文或者英文答题,不允许抄袭,一旦发现期末成绩计0分。

% 作业提交截止时间:12月25日晚上11点
\section{理性预期对有效市场的作用机制}
有效市场和理性预期假说是现代金融学理论的最重要基石。有效市场的假说是建立在市场中投资者完全理性的基础上的。
\end{document}