\documentclass{xjtureport}
\usepackage{color}
\usepackage[colorlinks,linkcolor=blue]{hyperref}
% \usepackage{enumerate}
% \newCJKfontfamily\sonti{FZCuSong-B09S}[BoldFont=FZCuSong-B09S]
% =============================================
% Part 0 Edit the info
% =============================================
% \titleformat{\section}{\bfseries\sonti\sectionef}{\thesection}{1em}
\major{大数据管理与应用}
\name{张锦羽}
\title{日本迅销集团下优衣库品牌的营销活动分析}
\stuid{2204112387}
\college{管理学院}
\date{\today}
% \lab{寝室}
\course{市场营销}
\instructor{宫翔}
% \grades{59}
\expname{市场营销第二次报告}% 副标题
% \exptype{设计实验}
% \partner{Bob}
% \setbeamertemplate{footline}[page number]
\setlength{\abovecaptionskip}{0.cm}
\setCJKmainfont{simsun.ttc}[ BoldFont=simhei.ttf,ItalicFont=simkai.ttf]
\begin{document}
% =============================================
% Part 1 Header
% ============================================={不想要封面就注释掉}============
\makecover
% \makeheader
\zihao{-4}

% =============================================
% Part 2 Main document
% =============================================
% \maketitle
\thispagestyle{fancy}

\rfoot{\thepage}
\section{研究背景}
\subsection{迅销集团(Fasting Retailing)简介}
迅销集团\cite{ref1}是日本的零售控股公司,也是世界第三大休闲服装公司,同时也是亚洲最大的服装公司。持有的品牌包括知名的\textbf{UNIQLO(优衣库)},以及ASPESI、Comptoir des Cotonniers、Foot Park、National Standard,GU等。FAST为“迅速”之意,而RETAILING即“零售”,这两者组合在一起,代表着像快餐业般步伐迅速的零售业。
\begin{figure}[H]
    \centering
    \subfigure[迅销集团logo]{
        \includegraphics[width=0.4\textwidth]{figure/FAST_RETAILING_logo.png}
    }
    \subfigure[迅销集团旗下品牌]{
        \includegraphics[width=0.5\textwidth]{figure/迅销旗下品牌.jpg}
    }
    \caption{迅销集团简介}
\end{figure}
\subsection{优衣库(UNIQLO)简介}
优衣库(日语:ユニクロ)是经营休闲、运动服装设计、制造和零售的日本公司,由柳井正创立,建立于1984年,是日本著名的休闲品牌,隶属于排名全球服饰零售行业前列的日本迅销公司旗下,在2005年11月1日公司重整后,UNIQLO现在为迅销的100\%全资附属公司,1999年2月该公司股票在东京证券交易所第一部上市\cite{ref2}。 并以预托证券形式在香港第二上市,2014年3月5日正式在香港挂牌上市。
% \begin{figure}[H]
%         \centering
%         \includegraphics[width=0.35\textwidth]{figure/优衣库logo.jpg}
%         \caption{优衣库logo}
% \end{figure}


定位为“快时尚”的UNIQLO,目前为全球知名服装品牌之一,除了日本之外目前在全球十七个地区展开业务。在亚洲地区,已经在日本、中国大陆、中国台湾、中国香港、韩国等东亚、东南亚国家和地区开设门店超过1500家。现任董事长兼总经理柳井正,在日本首次引进了大卖场式的服装销售方式,通过独特的商品策划、开发和销售体系来实现店铺运作的低成本化,由此引发了优衣库的热卖潮。在2018世界品牌500强排行榜中,优衣库排名第168位。
% \begin{figure}[H]
%     \centering
%     \includegraphics[width=0.7\textwidth]{figure/香港优衣库门店.jpg}
%     \caption{位于香港海港城内的优衣库店面}
% \end{figure}
\subsection{研究目的与意义}
优衣库作为亚洲最知名的服装快销品牌之一,在不同国家和地区都取得了商业上的巨大成功。诸如优衣库等这些快时尚品牌通过运用独特的营销策略,给我国消费者带来了不一样的消费体验,使我国的服装品牌在与其竞争过程中处于劣势地位。作为快时尚服装品牌中的姣姣者,优衣库在我国开店总数最多,其能够在中国发展迅速,并取得可观的经济效益,这与其别具一格的营销策略分不开。本文将利用麦卡锡的4P模型分析优衣库的营销管理活动(包括产品,定价,分销,促销四个方面)。
\section{产品角度:优衣库系列产品分析}
\subsection{产品层次}
产品层次,\textbf{指从满足客户需要的角度对产品所提供的服务进行的划分}。作为迅销集团的核心品牌,优衣库在面对同档价位的服饰品牌时具有强大的竞争力。这与优衣库有着明确的产品层次分类有很大关系。
% \begin{figure}[H]
%     \centering
%     \includegraphics[width=0.7\textwidth]{figure/产品层次.png}
%     \caption{产品层次图}
% \end{figure}
\begin{itemize}
    \item \textbf{核心利益:穿着}\\优衣库产品属于批发和零售业——零售业——纺织、服装及日用品行业,优衣库最基本的产品层次即是消费者在购买服饰类产品时所追求的最核心的利益——穿着。所有服饰行业的产品都会优先满足消费者穿着的需求,否则将无法在服饰业立足。
    \item \textbf{基础产品:穿着的具象化,如衣服、裤子等}
    
    由于优衣库倡导的百搭理念,所有优衣库销售的服装都是一些好搭配的基础款,这些基础款的服饰几乎涵盖了所有顾客生活中可能会穿到的休闲单品,而且这些基本款的品类、尺码齐全、产品种类多样并且百搭,也满足了消费者日常穿着的需求。
    \item \textbf{期望产品:面料选择、抗寒保暖透气等方面}。可以说优衣库的服饰基本能够满足人们对理想服饰的大部分需求。
    \begin{enumerate}
        % \item 首先是\textbf{性价比方面}:营销领域的性价比定义为产品性能与产品价格的比值:$$\text{性价比}=\frac{\text{产品性能}}{\text{产品价格}}$$
        % \begin{enumerate}
        %     \item \textbf{在产品性能方面},大中华区一半以上的优衣库服饰都由位于江苏省常州市的晨风工厂进行代工,中国大陆地区消费者平常购买的优衣库服饰大多出自这里。同时晨风工厂的第一大客户也是优衣库,两者之间建立了很亲密的合作关系。本人在2022年度荣获“第十届UNIQLO奖学金”,在2022年暑期被优衣库邀请参观位于江苏省苏州市的晨风工厂\cite{ref4}。优衣库衣服制作的工艺非常严谨,晨风集团董事长尹国新先生向我们讲述了,为优衣库制作服饰的优质品与残次品的标准:“领口、袖口等缝合处多余2处线头即为残次品,需要工人们进行返工。”所以优衣库的服饰已经将质量作为一大卖点,声称“穿三年也不会坏”。
        %     \begin{figure}[H]
        %         \centering
        %         \subfigure[参观晨风工厂]{
        %             \includegraphics[width=0.45\textwidth]{figure/参观晨风.jpg}
        %         }
        %         \subfigure[我本人在体验优衣库服饰制作]{
        %             \includegraphics[width=0.45\textwidth]{figure/我的体验.jpg}
        %         }
        %         \caption{产品性能方面}
        %     \end{figure}
        %     \item 在产品价格方面,优衣库聚焦于基础款休闲服饰,在价格上极具竞争力。优衣库产品的价格普遍集中于人民币39元至499元之间。我将在{\color{red}“产品定价”}板块详细说明。
        % \end{enumerate}
        \item 一是\textbf{面料选择方面}:优衣库深信“真正优质的服装可以改变世界”。所谓优质的服装,是指任何人都能够感知其价值的服装以及所有人都希望穿着的服装。\cite{ref6}优衣库坚持在全球范围内精选上乘面料、开发新功能面料创造消费新需求。在面料选择方面,优衣库具有两大优势:
        \begin{enumerate}
            \item 能够以亲民的价格,为人
            们提供使用羊绒、SUPIMA棉、优质羽绒等高档原
            料制作的商品。譬如,动辄售价上千元的羊绒衫,在优衣
            库的店铺里却只要699元(男装V领羊绒衫)。多年来,
            优衣库始终通过与全球面料制造商直接洽谈、大
            批量采购原材料等方式,维持价格优势。
            \item 优衣库的优势在于:以任何人都能够承受的
            价格,提供前所未有的新功能服装,为顾客创造新
            价值。优衣库携手日本大型合成纤维制造商东丽
            公司共同研发,耗时多年,不断改进,终于开发出
            HEATTECH等新功能面料及令人满意的相关商品。
        \end{enumerate}
        \item 二是\textbf{抗寒保暖透气方面},前面提到了优衣库对优质面料的追求和极致性价比。优衣库被快时尚品牌界定义为技术控,因为它从始至终都在热衷一件其他快时尚品牌不会关心的事——研发。受益于严谨的研发理念,期保暖服饰系列产品“Heattech”在世界范围内都取得了巨大成功,几乎引领了轻薄保暖时代的来临。除此之外,保湿系列、快干系列、轻羽绒系列产品也已经形成特色,成为优衣库系列产品的闪亮名牌。
    \end{enumerate}
    \item \textbf{附加产品:优衣库的时尚潮流属性}\\
    优衣库虽然属于零售快销行业,但是与其他快时尚品牌一样,优衣库也非常注重服饰的时尚性。优衣库在东京、纽约、巴黎、米兰四大时尚之都都设置了产品研发中心,搜集当地奢侈品牌、店面以及四大时装周传递出的时尚资讯,形成对当季度流行趋势的判断。四个产品中心同时设计,并根据各国市场特性调整产品组合。为保证优衣库时尚潮流系列符合“轻奢”路线,单季度只推出350-400款产品,只有其他休闲服饰品牌的1/5至1/3。\cite{ref7}
%     \item \textbf{附加产品:便捷的门店选购、完善的售后服务、线上线下双渠道}
    
%   

%     另外优衣库秉持独特的经营理念——“所有人都能穿的休闲服”。优衣库在1984年成立之初将自
% 己的风格定位于“百搭- Made for All”、“服装的零配件”,宗旨在于使所有人
% 都能穿上高品质休闲服。仓储型店铺,超市型
% 自主购物。优衣库店铺从选址到装修贯彻的理念就是“让顾客可以自由选择的环境”,
% 几乎都是“大店”风格,宽敞如仓库,根据消费者的消费习惯合理布局。同时全球优衣库店铺基本如同服装仓库,在优衣库门店可以买到优衣库当季推出的所有类型服饰,很少会出现种类短缺的问题。优衣库门店的设计很大减少了消费者选择服饰过程中的时间成本、提高了顾客的体验价值。
    \item \textbf{潜在产品:“零件化”服饰产品、“LifeWear”生活态度}
    \\优衣库非常注重消费者在穿着服饰时的体验感,为此优衣库的企业管理人员提出(服适人生)“LifeWear”的产品设计哲学,现已经成为优衣库的企业文化的一部分。
        
    “LifeWear”是让所有人的生活更加丰富多彩的服裝。它具有美学的合理性,简单、高品质且追求细节的完美, 是基于生活需求而设计、并不断发展不断进化的日常服。优衣库始终坚持将大多数日常服装定位为大众、价格便宜、质量优秀的同时又追求潮流感和时代感是优衣库实现品牌差异化的重点。未来优衣库可能会将产品视角更多从企业转移到消费者,考虑将休闲服饰“零件化”,将更多的选择权交给消费者,继续引领休闲服饰变革。
\end{itemize}
\subsection{产品差异化战略}
\subsubsection{优衣库产品现有战略}
\begin{enumerate}
    \item 基于顾客价值的差异化战略
    
\begin{enumerate}
    \item 产品价值:根据马斯洛层次需要理论分析优衣库带来的产品价值,以及其据此制定的
产品差异化战略。
    \begin{itemize}
        \item \textbf{生理需要:}\\
        首先是生理需要,一件服饰除去消费者的偏好与其时尚与潮流后,就是需要满足人穿着衣服的基本需求。如春夏透气凉爽,秋冬保暖抗寒。由于优衣库在世界范围内严选优质面料,同时开发新型面料。每件新产品上线需要经过“确定概念——确定面料——制作样品——开发与采购面料——确定款式”的流程。\cite{ref6}只有迎合消费者最根本的生理需要,才有可能进入市场。
        \item \textbf{安全需要:}\\
        其次是安全需求,优衣库有对衣服面料的层层把关机制,确保贴身衣物面料纯天然,顾客不必担心化学成分对皮肤组织的侵蚀作用。此外,在新冠肺炎肆虐的背景下,优衣库也推出了口罩系列产品与口罩联名系列产品,确保消费者的安全需要。防范新冠病毒的感染,保护消费者身体健康。
        \begin{figure}[H]
            \centering
            \includegraphics[width=0.8\textwidth]{figure/U口罩.png}
            \caption{优衣库首页的口罩广告页面}
        \end{figure}
        \item \textbf{社交需要:}\\
        第三是社交需要,优衣库的线下售卖方式很好的给到了顾客群体社交体验。顾客在优衣库门店进行选购的过程中,优衣库的工作人员会持续陪同选购,同时介绍产品的特性与售卖情况,顾客可以与陪同的优衣库员工进行亲切交流,优衣库员工也有高素质的训练与严格的考评标准,即使只挑选不购买,也可以示意优衣库工作人员不要打扰,工作人员也会回以微笑然后默默离开。同时优衣库也会经营粉丝社群,购买者可以在社区内分享自己购买产品的评测和穿着体验。打造品牌粉丝效应。
        \item \textbf{自尊需要:}\\
        优衣库的衣服不同于其他服饰,其品牌定位就是“用于匹配其他大牌衣服的服装。”其全系都没有在外可见的logo,大部分都是纯色内衣和基本款牛仔裤就是为了让消费者可以随意的去与其他服装搭配。其百搭的“零件式”产品风格广受消费者喜爱。即使不怎么会穿搭的消费者,也能通过优衣库线下门店或是线上选购的方式获得购买建议,购买到自己满意的服饰。
        \item \textbf{自我实现需要:}\\
        优衣库向消费者传达“lifewear”的服装设计哲学。提高了消费者的自我认同能力与感知能力。同时优衣库也致力于推广服装的“永续发展”理念,在2014年,UNIQLO开始为衣物回收、再生再利用做号召。台湾优衣库海外行销暨宣传战略部部长黄佳莹:“希望回收的衣物可以被充分使用,因此他们与台湾采印协会合作。同时关注到偏乡资源分配不均的状况,可以把服装送给孩子们。”\cite{ref9}。消费者在购买优衣库服饰的过程中给消费者一种回馈社会的参与感,给予消费者自我实现的感受。
        \begin{figure}[H]
            \centering
            \subfigure[优衣库永续发展]{
                \includegraphics[width=0.45\textwidth]{figure/优衣库永续发展.png}
            }
            \subfigure[作为UNIQLO全球永续发展大使的绿色哆啦A梦]{
                \includegraphics[width=0.45\textwidth]{figure/绿色dola.jpg}
            }
        \end{figure}
    \end{itemize}
    \item 服务价值
    \begin{itemize}
        \item 门店选购:导购、人性化、种类齐全
        
    优衣库主要采用的是线下门店的销售制度。“所有人都能穿的休闲服”。优衣库在1984年成立之初将自
己的风格定位于“百搭——Made for All”、“服装的零配件”,宗旨在于使所有人
都能穿上高品质休闲服。

优衣库店铺从选址到装修贯彻的理念就是“让顾客可以自由选择的环境”,
几乎都是“大店”风格,宽敞如仓库,根据消费者的消费习惯合理布局。采用超级店长+店员共同管理,优衣库将权力下放至店长,一家店铺经营情况主要是由门店店长进行负责,所有不同于很多加盟店店长的“打工人”属性,优衣库店长具有高度的自治权,可以根据地区政策制定最适合顾客的打折促销模式,提供给当地顾客最人性化的服务类型(如地方方言导购、不同国家和地区的文化差异)。

同时优衣库门店都基本分为:男装、女装、UT、婴幼儿等多个区域,超市型
自主购物。同时全球优衣库店铺基本如同服装仓库,在优衣库门店可以买到优衣库当季推出的所有类型服饰,很少会出现种类短缺的问题。优衣库门店的设计很大减少了消费者选择服饰过程中的时间成本、提高了顾客的体验价值。
        \item 退换货政策宽松,充分考虑消费者权益
        
        由于优衣库本身的零售性质,其产品的销售主要是通过官方线下门店和官方网上门店的两种方式进行分销。本身减少了商品的流通环节,由自己作为批发商和分销商,所以消去了中间商的差价部分。一定程度上减少了服饰的批发成本。同时优衣库规定:商品在未经穿着、洗涤、损坏并保留吊牌和洗涤的前提下,在购买后1个月内可退换。 非顾客原因造成质量问题的商品,3个月内可无条件退换。\cite{ref8}充分保障消费者权益。
    \end{itemize}
    \item 人员价值
    \begin{itemize}
        \item 代言人价值:充分考虑地域差异,采用地区具有影响力的代言人\\
        日本:栗山千明、前田敦子、新垣结衣、松田龙平、黑木明纱、水原希子、本田圭佑、锦织圭、长濑智也、藤原纪香、中谷美纪\\
        代言中国大陆地区:陈坤、孙俪、杜鹃、黄豆豆、谭元元、高圆圆、倪妮\\代言中国香港地区:方大同、黄宗泽\\代言中国台湾地区:陈柏霖、陈意涵、严爵、林依晨\\韩国:bigbang、 姜东元、全智贤\\可以看出优衣库在挑选代言人时并不一味追求流量,而是选用适合产品,符合地区消费者口味的代言人,充分利用代言人人员价值。
        \item 优衣库店员:严格考核,温柔相待,发掘店员价值:\\
        作为和用户群体接触最多的店员,优衣库对店员的考核标准时严格的,“看到客人要微笑”、“耐心讲解”等标准赫然被纳入量化考核指标。尽管有着严格的考核标准,优衣库也有温情待人的一面,每天休息的时候会让整个店铺的员工一起写Thank U card,增强内部凝聚力。
        \item 优衣库店长:UMC制度,严格考评制度。\\
        优衣库内部的考评制度也非常严格,普通员工晋升店长至少需要五次升职,每次至少三轮考核与面试,同时优衣库还有UMC制度,每年从校招中的优秀大学生中选择优质人才进行管培生教育并出任店长。
    \end{itemize}
    \item 形象价值
    \begin{itemize}
        \item 社会公益:提升品牌形象\\优衣库积极打造品牌形象。投身社会公益,开展“永续发展”、“优衣库温暖周”等社会公益活动,为防疫志愿者\cite{ref10}、战争难民、贫困山区、地震灾区的难民捐赠超过500万件衣服、累计捐赠超过1000万元。
        \item 与基金会合作、开设奖学金\cite{ref4}。在年轻一代人心中树立形象\\
        优衣库与中国教育国际交流协会,缔结了优衣库奖学金项目。面向中国(港澳台地区除外)教育部直属大学的本科在读学生,提供一定金额的奖学金以及国际交流机会,旨在培养未来的国际化人才,为中国青年人活跃在世界舞台做出贡献。我有幸获得了“第十届UNIQLO奖学金”,并已经参与了上海的交流活动,优衣库这家企业已经在像我一样的456名获得奖项的大学生心中树立了榜样。
            % \begin{figure}[H]
            %     \centering
            %     \subfigure[参观晨风工厂]{
            %         \includegraphics[width=0.45\textwidth]{figure/参观晨风.jpg}
            %     }
            %     \subfigure[我本人在体验优衣库服饰制作]{
            %         \includegraphics[width=0.45\textwidth]{figure/我的体验.jpg}
            %     }
            %     \caption{优衣库奖学金\cite{ref4}}
            % \end{figure}
    \end{itemize}
    \item 体验价值\\
    无论是“LifeWear”、“永续发展”还是“优衣库温暖周”。优衣库总是能让消费者产生强烈的社会责任感与社会认同感,优衣库试图使用自己独有的价值观与消费者对话,潜移默化影响消费者。消费者在购买优衣库服饰的过程中也能收获一种回馈社会的幸福感与满足感。
\end{enumerate}
    \item 基于产品层次的差异化战略\\
    同样是服饰型产品,优衣库与其他服饰产品的最大购买点在于其\textbf{期望产品、附加产品与潜在产品}。优衣库在面料选择、抗寒保暖等属性上显著优于普通服饰产品,并相较于同等价位的服饰产品增添了时尚与潮流的属性。同时优衣库也在其企业影响力向行业和消费者传达“LifeWear”的服饰设计哲学,引领休闲服饰行业变革。
    \item 基于产品构成要素的差异化战略
    \begin{itemize}
        \item 规格型号差异化\\
        优衣库产品的规格型号多样,按照规格可以分为:UT系列(优衣库T恤)、长袖连体装、外套、背心、运动衫、睡衣、袜子等;按照型号又可分为各种衣服尺码,如XS、S、M、L、XL、XXL$\ldots$。
        \begin{figure}[H]
            \begin{center}
            \includegraphics[width=0.55\textwidth]{figure/优衣库女装.png}
            \end{center}
            \caption{优衣库官网产品系列\cite{ref11}}
        \end{figure}
        \item 结构外观差异化\\
        优衣库产品提供老少皆宜的多种服饰选择,可以分为男装、女装、童装、婴幼儿装等。每种服饰类型各有面料选择和服饰我外观的特色。
        \item 品牌商标差异化\\
        图\ref{优衣库商标}展示了优衣库的商标,其一,LOGO设计为四方形,并且在四方形中“UNIQLO”字体几乎占据了全部的比例,来表达优衣库的自信与大气。其二,设计师将LOGO的底色设计成酒红色。意为“优衣库是成衣界的日本代表,所以用上日本国旗‘日之丸’的红色,可以彰显优衣库的存在价值”。其三,在大中华区“UNIQLO”品牌被翻译为“优衣库”,体现了翻译过程中的“信、达、雅”特点,“优衣”代表了优衣库服饰卓越的品质,“库”意指服饰种类的齐全,体现了优衣库线下门店“仓储式”、“超市型”购物的特点。
        \begin{figure}[H]
                    \label{优衣库商标}
                    \centering
                    \includegraphics[width=0.4\textwidth]{figure/优衣库logo.jpg}
                    \caption{优衣库logo}
        \end{figure}
        \item 产品包装差异化\\
        优衣库产品在购买后会有独立的精美购物袋和精美的封贴胶布进行包装,购物袋材质采用硬纸质,给予消费者高质量的购物体验;同时考虑到环境因素,胶布采用可降解材料,承担社会责任。
        \item 产品服务差异化\\
        优衣库产品的附加服务主要包括两大类:线下的门店导购选购与线上的在线客服交流。在线下门店,会有专人耐心讲解帮助顾客选购,在线上旗舰店也会有在线客服解答消费者疑问。同时优衣库的退换货政策持续到消费者购买衣服起之后的60天时间范围内\cite{ref8}。无论是线上线下,还是退换货政策,都给予了消费者安心的购物体验。
    \end{itemize}
\end{enumerate}
\subsubsection{未来的改进方向}
\begin{enumerate}
    \item 根据门店地理位置特点,因地制宜制定营销策略:\\
    优衣库门店遍布五大洲的国家和地区。区域文化、习俗、宗教等差异极大。即使在同一国家内,也要考虑不同区域消费者的风俗习惯,如中国的南北方差异和美国的东西部差异等。诸如在一些新兴的市场,如中东地区(沙特、阿联酋等国家和地区)要充分调研当地消费者日常穿搭需求,考虑宗教因素,认识到当地女装与欧美国家之间的差异、削减产品线。
    \item 完善并持续发展线上购买渠道,完善app端功能\\
    在前疫情时代,优衣库的售卖方式是通过线下门店的方式。然而随着大陆地区防疫政策与疫情的冲击,越来越多的消费者转向线上购物,这对以线下门店为主要销售渠道的优衣库来说是不小的挑战。如何快速转型,把握消费者线上线下行为的不同与相似之处,如何让消费者在线上也能感受到优衣库的热情服务态度,是优衣库未来需要不断完善与探索的。优衣库线上销售渠道app等需要进一步的优化。根据用户评论可以得出,优衣库app还存在着诸如“UI界面差”、“卡顿”、“闪退”等缺点,给消费者造成了不好的购物体验;
    优衣库现在可以通过手机app端向顾客推送新品资讯,用户也可以直接通过微信小程序的方式下单优衣库系列商品。优衣库可以利用信息时代的技术,通过app端的用户行为数据进行用户行为分析进行推销。这一点优衣库相比于依托互联网公司的大型电商平台,如淘宝、京东、天猫等做的还不够好。未来将进一步优化其网络营销端的算法,进行用户群体的分类与精准营销。
    \item 代工厂的服饰质量要求一致,严格把关\\
    优衣库服饰以其高质量低价格的特点立足于消费市场,优衣库在世界范围内拥有60多家代工厂\cite{ref6},就以大中华区的主要代工厂之一,晨风工厂为例:大中华区一半以上的优衣库服饰都由位于江苏省常州市的晨风工厂进行代工,中国大陆地区消费者平常购买的优衣库服饰大多出自这里。同时晨风工厂的第一大客户也是优衣库,两者之间建立了很亲密的合作关系。本人在2022年暑期被优衣库邀请参观位于江苏省常州市的晨风工厂\cite{ref4}。优衣库衣服制作的工艺非常严谨,晨风集团董事长尹国新先生向我们讲述了,为优衣库制作服饰的优质品与残次品的标准:“领口、袖口等缝合处多余2处线头即为残次品,需要工人们进行返工。”所以优衣库的服饰已经将质量作为一大卖点,声称“穿三年也不会坏”。然而并不是所有优衣库的服饰都有着晨风工厂一样的严格要求,比如优衣库也会遭遇假货假冒山寨品牌的威胁。\cite{ref12}
        \begin{figure}[H]
            \centering
            \subfigure[参观晨风工厂]{
                \includegraphics[width=0.45\textwidth]{figure/参观晨风.jpg}
            }
            \subfigure[我本人在体验优衣库服饰制作]{
                \includegraphics[width=0.45\textwidth]{figure/我的体验.jpg}}
            \caption{优衣库代工厂——晨风工厂调研}
        \end{figure}
\end{enumerate}
\subsection{优衣库产品延伸战略}
\subsubsection{现有战略}
\begin{enumerate}
    \item \textbf{向上延伸}\\
    优衣库的向上延伸,采用运营化、本土化、场景化的发展战略,助力高端服饰品牌推广,来满足高端定位人群的消费需求,由市场专业化向市场全面化扩散(从低档定位到高档定位)。
    \begin{itemize}
        \item \textbf{运营化:}优衣库定位为“平价休闲服装”,同时优衣库服饰又自带“快时尚”属性,同时迅销公司旗下的“GU”系列产品就是定位为高端服饰市场,依托“GU”品牌开拓的市场,优衣库接连推出的高端产品、科技型产品也都取得了成功。同时优衣库依托自己的线下门店和网络旗舰店,在店面的显眼位置或者网站首页进行高端产品的展示,提高产品的曝光度,逐渐向高端定位渗透。
        \item \textbf{本土化:}同时优衣库在东京、纽约、巴黎、米兰四大时尚之都都设置了产品研发中心,搜集当地奢侈品牌、店面以及四大时装周传递出的时尚资讯,拥有向上延伸的野心。
        \item \textbf{场景化:}优衣库推出的高档服饰主要针对一些特定的使用场景,如酒会、典礼等。并且融合优衣库休闲服饰中的科技,使得西装具有透气轻薄的特点,与其他类型产品相比具有很大的竞争优势。
    \end{itemize}
    
    \item \textbf{平行延申}\\
    优衣库在平价休闲服装领域推陈出新,推出的“UT系列”产品,诸多“联名”产品更是一件难求。充分在年轻人群体,潮流爱好者心中树立了品牌形象。曾经优衣库与KAWS的联名爆款让优衣库联名系列产品第一次出圈,大大提高了店铺知名度\cite{ref13}。从此“UT系列”产品成为了优衣库最出圈的系列产品。一件不足百元的T恤为何会遭到年轻人的疯抢?优衣库在平行延伸的过程中,根据休闲服装市场上的空白点进行延伸,提出了其它同
    类产品所没有的联名款,使得其具有较高的盈利能力和竞争地位,企
    业的产品线也得到了优化。在休闲服装领域市场中优衣库与其他品牌做到了差异化竞争,通过“联名”、“跨界”和“限定”等营销方式拓宽普通的休闲服装市场,成为行业巨头。
    \begin{figure}[H]
        \centering
        \includegraphics[width=0.6\textwidth]{figure/联名.jpeg}
        \caption{优衣库与KAWS最后一次联名UT广告图}
    \end{figure}
\end{enumerate}
\subsubsection{未来改进方向}
\begin{enumerate}
    \item 针对三四线城市做下沉市场布置工作\\
    根据优衣库线下门店规划,中国大陆地区只有一二线城市和少数三线城市拥有优衣库门店,且门店比较集中。比如仅上海一个直辖市就拥有44家门店,而整个河北省也仅仅拥有门店7家。这说明优衣库门店的布置具有不平衡的特点。而中国大陆地区三四线城市人口约占73\%\cite{ref14},拥有巨大的发展潜力。优衣库创始人柳井正先生在接受《财富》(中文版)专访时曾表示,他期望中国店铺数量达到3000家。\cite{ref15}
    \item 依托迅销旗下“GU”品牌影响力,为优衣库向上延伸创造条件和机遇:
    \\优衣库在向上延申的过程中收到了许多消费者的阻挠,“优衣库变质了”,“开始圈消费者钱”等言论一时在优衣库社区内层出不穷。优衣库在向高端服饰进军的过程中必然会收到消费者的不理解等阻挠,也可能会承受消费者的流失。可以利用GU的名牌效应助力优衣库向高端服饰产品转型。可以先推出一些GU \& UNIQLO的联名产品,吸引人气,在逐渐推出设计更加精巧,品质更加优越的服饰产品。过程应该循序渐进,而不应该一蹴而就,这样只会对消费者造成伤害。
\end{enumerate}
\clearpage
\section{优衣库定价策略}

\subsection{需求导向定价法}
优衣库根据消费者对服饰的需求强度不同,以不同时间季节、产品样式为基础进行差别定价,能够在不引起消费者反感的情况下,获取最大利润。
\begin{enumerate}
    \item 时间季节差异\\
    服饰品产业对于季节变化是非常敏感的,消费者对于优衣库服饰的需求并不是全年保持不变的,几乎在世界各地的优衣库门店都会出现淡季与旺季。以2014年的旧金山优衣库门店为例,与12-2月的冬季月份相比,7-9月的夏季月份需求要高得多(UT的价格优势与名牌效应)。每个地区的具体高点和低点都是不同的。
    \item 地域位置差异\\
    不同地域的人们的需求是不同的。从大的方面来说,各个国家和地区都有自己当地的基本情况,各个国家的消费者消费水平也不同,海关关税和货币汇率等因素也是优衣库在产品定价时需要考虑的。
    
    另一方面,地域不同导致各个地区风俗习惯不同,可以根据当地节日,习俗等制定促销打折的策略,吸引消费者。
    \item 产品样式差异\\
    即使同一件优衣库产品,拥有同样的外观,也以因为面料、填充物、配件的不同而制定不同的价格
\end{enumerate}
\subsection{竞争导向定价法}
\subsubsection{渗透定价策略}
渗透定价策略是指优衣库产品在早期上市时就将价格定在十分
亲民的水平,从而迅速吸引消费者,提早占领市场。优衣库在品牌定位初期就主打平价休闲服饰,相对于一些没有品牌优势的国货和相对高端的进口服饰产品来说,优衣库具有独特的价格优势。例如,上文提到的与优衣库联名
的KAWS品牌属于艺术家风格品牌,其高昂的定价一直以来都使绝大多数年轻人望而
止步,而“UNIQLO $\times$ KAWS”的联名T恤,其上市价格仅仅99元,\cite{ref13}广大年轻人长期求而不得的强烈差距,在联名款T恤发售时被迅速填补,从而创下了销售奇迹。99元的价格被粉丝戏称为“可能是唯一一次能卖得起KAWS作品的机会了。”\cite{ref16}
\subsubsection{价格调整策略}
价格调整策略是指优衣库在热卖期抢先降价,同时将一些质量上乘但是断码或
有一点小瑕疵的服装进行降价,不仅减少库存还能吸引消费者。通过
一系列的现金折扣和数量折扣的方式,在消费者购买产品时进行促销,从而提高产品
的销量和企业的效益。优衣库的这些定价行为更好的满足了消费者对于服装价格的
期望,得到了消费者的认可和喜爱。
\section{优衣库分销策略}
\subsection{直接分销}
优衣库采用直接分销的方式,无需经过分销商即可直接把产品销售给顾客。尽
管直接分销的渠道短、覆盖面窄,但是也使得优衣库能直接与顾客沟通,联络感情。并
且,通过社交平台的直接分销,客户也会具有较高的情感认同与黏性。具体来说,直
接分销有以下四种方式。
\begin{enumerate}
    \item 建立大中华区业务团队搭建直销渠道,消费者直接在官方旗舰店上交易\\
    优衣库通过设立大中华区业务团队进行本土化运营和管理,包括设立大中华区专业数
据团队、运营团队,以及全天在线的国内客服团队,来保障服饰的销售、宣传与
管理。优衣库网上旗舰店UI设计简约却人性化,界面设计中能够抓住消费者眼球,具有列表图片
切换、列表实时更新、搜索结果保存等人性化功能。
\item 零级渠道:制造并销售\\
零级渠道即一种由制造商直接到消费者的分销渠道。
\item 代工厂$\to$线下门店:通过一级渠道(MRC)进行销售:
    MRC即由制造商——零售商——消费者的分销渠道。优衣库的大部分服饰是由不同的代工厂进行加工制造的。由工厂加工成品后直接运输到各个线下门店进行销售。
    \item 针对会员用户或者目标用户直接发送邮件推销\\
    优衣库app的注册会员、在线下门店购买过优衣库的用户或是通过微信小程序等购买过优衣库产品的用户,优衣库都能通过顾客预留的联系方式向用户发送邮件、短信、消息提醒等进行推销。
    \item 官方账号利用社交平台进行推广分销\\
    优衣库积极构建粉丝社群,构建官方粉丝俱乐部,分享粉丝购物体验等,利用社交平台的传播性,用户感受的真实性等因素进行优衣库服饰的推广与促销。
\end{enumerate}

\clearpage
\begin{thebibliography}{99}  
\bibitem{ref1} \href{https://web.archive.org/web/20210124040046/https://www.fastretailing.com/jp/about/company/}{迅銷集團. 概况 FAST RETAILING CO., LTD.. [2018-04-26]}. 
\bibitem{ref2}  \href{https://www.fastretailing.com/jp/about/links/}{リンク集 | FAST RETAILING CO., LTD.}
\bibitem{ref3} \href{https://baike.baidu.com/reference/4397542/9ccb-0QL1Cbyq9azCzh6JA2suTV_kom-CcDBaAEnefnbjvKldmwqHsy9q1C5egTSJDXW25uOpkB2khESK5zemczKZuGH0pJm_SvCzphrKaB9MVG5-A}{世界500强   .世界品牌实验室[引用日期2018-12-20]}
\bibitem{ref4} \href{https://mp.weixin.qq.com/s/D7sGPq8OxnwvS7fyHyUTMg}{恰同学少年:2022优衣库奖学金授予仪式苏州行}

\bibitem{ref5} \href{https://www.fastretailing.com/jp/sustainability/labor/pdf/FRCoreSewingFactoryList.pdf}{优衣库代工厂名单}

\bibitem{ref6} \href{https://www.fastretailing.com/tc/ir/library/pdf/ar2014_tc_07.pdf}{优衣库事业.pdf}
\bibitem{ref7} \href{https://kns.cnki.net/KCMS/detail/detail.aspx?dbname=CMFD201501&filename=1014374415.nh}{孙建国.(2014).优衣库营销策略研究(硕士学位论文,北京交通大学).}
\bibitem{ref8} \href{https://www.uniqlo.com/cn/corp/returnpolicy/}{优衣库退换货政策}
\bibitem{ref9}\href{https://www.wantshowlaundry.com/uniqlo/}{UNIQLO永续行动,延续美好生活的LIFEWEAR,服装循环再制让我们看见永续生活的各种可能性}
\bibitem{ref10} \href{https://socialbeta.com/c/9562}{优衣库携消费者为「城市英雄」送温暖}
\bibitem{ref11} \href{https://www.uniqlo.cn/}{首页 - 优衣库网络旗舰店}
\bibitem{ref12} \href{http://gftai.bcpcn.com/articles/525/49019.html}{优衣库被工商查出产品质量不合格 涉嫌欺诈}
\bibitem{ref13} \href{https://www.jiemian.com/article/3189175.html}{优衣库KAWS联名遭疯抢:从炫富到炫潮,一场99元的降维打击}
\bibitem{ref14} \href{https://finance.sina.com.cn/zl/china/2021-06-09/zl-ikqcfnaz9918291.shtml}{中国人口大迁移:2020}
\bibitem{ref15} \href{https://finance.sina.com.cn/chanjing/gsnews/2022-11-30/doc-imqqsmrp8140719.shtml}{优衣库创始人:在中国开3000家店还是不够}
\bibitem{ref16} \href{http://www.ccidanpo.org/sites/default/files/c26636204.2020.10.02.pdf}{黄洁. 优衣库在华营销策略分析及启示[J].2020.10.02}
\end{thebibliography}
\end{document}
